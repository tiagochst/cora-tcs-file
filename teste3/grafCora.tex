\documentclass[10pt,a4paper]{article}
\usepackage[T1]{fontenc}
\usepackage[brazil]{babel}
\usepackage[utf8]{inputenc}


\usepackage{ae,aecompl}
\usepackage{pslatex}
\usepackage{epsfig}
\usepackage{geometry}
\usepackage{url}
\usepackage{textcomp}
\usepackage{ae}
\usepackage{subfig}
\usepackage{indentfirst}
\usepackage{textcomp}
\usepackage{color}
\usepackage{setspace}
\usepackage{verbatim}
\usepackage{hyperref}

% Gráficos
\usepackage{pgfplots}
\pgfplotsset{compat=1.3}
\usepgfplotslibrary{groupplots}

\usepackage{hyperref}
\hypersetup{
    colorlinks,%
    citecolor=black,%
    filecolor=black,%
    linkcolor=blue,%
    urlcolor=blue
}

\usepackage[compact]{titlesec}
\titlespacing{\section}{0pt}{*0}{*0}
\titlespacing{\subsection}{0pt}{*0}{*0}
\titlespacing{\subsubsection}{0pt}{*0}{*0}

% Definindo as margens para 2cm e o espaçamento entre linhas para 1.5
% Relatório parcial deve ter espaçamento simples
% \linespread{1.5}

\geometry{ 
  a4paper,	% Formato do papel
  tmargin=30mm,	% Margem superior
  bmargin=30mm,	% Margem inferior
  lmargin=20mm,	% Margem esquerda
  rmargin=20mm,	% Margem direita
  footskip=10mm	% Espaço entre o rodapé e o fim do texto
}
\renewcommand{\thetable}{\Roman{table}}
\newcommand{\x} {$\bullet$}


\begin{document}

\vspace{-5mm}

\begin{tikzpicture}
  \begin{axis}[
    ylabel=taxa,
    xlabel=Tempo,
    height=14cm,
    width=14cm,
    grid=major,
    ytick = data,
ytick ={1, 2, 3, 4, 5, 6, 7, 8, 9, 10,11,12},
yticklabels={1,2,5.5,6,9,11,12,18,24,36,48,54},
legend style={at={(0.5,-0.15)},
anchor=north,legend columns=-1}]    
    \addplot[mesh]
    table[x=t,y=rate] {minstrelPlot};
  
    \legend{ Algoritmo minstrel }
  \end{axis}
\end{tikzpicture}

\centering
\begin{tabular}{lllll}
TAXA & Throuput EWMA & ENVIADOS & TENTATIVAS & Porcentagem   \tabularnewline
\hline
1 & 0.8 & 44 & 48 & 91.67\% \tabularnewline
2 & 0.4 & 1 & 1 & 100.00\%  \tabularnewline
5.5 & 1.2 & 1 & 1 & 100.00\% \tabularnewline
6 & 0 & 0 & 0 & 0.00\% \tabularnewline
9 & 0 & 0 & 0 & 0.00\% \tabularnewline
11 & 0 & 0 & 0 & 0.00\% \tabularnewline
12 & 0 & 0 & 0 & 0.00\% \tabularnewline
18 & 0 & 0 & 0 & 0.00\% \tabularnewline
24 & 21 & 200 & 202 & 99.01\% \tabularnewline
36 & 13.3 & 2 & 2 & 100.00\% \tabularnewline
48 & 17 & 2 & 2 & 100.00\% \tabularnewline
54 & 43 & 8608 & 8712 & 98.81\% \tabularnewline
\end{tabular}
\begin{tabular}{llll}
 média de throughput & total enviado & total tentativas & total enviado/tentativas   \tabularnewline
\hline
42.2567573595004 & 8858 & 8968 & 98.77\% \tabularnewline
\end{tabular}


\end{document}
